%! TEX root = ./main.tex

\section{Conclusion}

Forests have a critical impact on global climate change. Therefore, the carbon sequestration capacity of forests should receive focused attention. We developed a carbon sequestration models and a Multi-aspect Decision Model of Forest Management using various methods such as biomass method, entropy weight method, linear fitting method, and cluster analysis method for estimating the carbon sequestration of forests and the comprehensive value of forests. We selected the forest land in Hunan Province as an example analysis, and presumed that its carbon sequestration under the existing management strategy is 160.6 ton. Meanwhile, we proposed to adjust the logging rate to 0.01 and the production strategy to maximize the production of sawn timber, taking into account its natural condition. A smooth transition model was also proposed as a transition strategy based on the change of cutting period. A science-based news article was eventually written to disseminate the study results and provide solutions for community forest management strategies.