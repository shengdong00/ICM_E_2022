%! TEX root = ./main.tex

\newpage

\section{Appendixes}

\subsection{A: Codes}
Below is the source code for carbon sequestration model. Since all three species share the same algorithm, we only represent the source codes for $Choerospondias$.
\begin{lstlisting}
function [CT,CP,C,x] = Task(p,choice,year)
    j = sym('j');
    t = sym('t');
    mu1 = 0.5259; eta1 = 0.1764; rho1 = 0.596;
    average_D1 = 16.86-1.25; N1 = 12600;
    T1 = 35; p1 = p; P1 = 40+year;
    q = 0.334; r = 0.0912;
    Kc = 0.5;
    L1 = 15; Pr1 = 5300;
    if choice == 1
        alpha1 = 0.8*mu1; beta1 = 0; gamma1 = 0.2*mu1+eta1;
    else
        alpha1 = 0; beta1 = 0.8*mu1; gamma1 = 0.2*mu1+eta1;
    end
    
    fit_func = fittype('c/(1+a*exp(-b*t))', 'independent', 't',
    'coefficients', {'a','b','c'});
    raw_D1 = [0 4.650302778 9.912501403 13.17246799 14.9256427
    15.89324511 16.69899162 17.081377]';
    j1 = [0:5:35]';
    cfun_D1 = fit(j1, raw_D1, fit_func);

    raw_H1 = [0 6.21600482 12.83876406 14.10135112 14.79148078
    14.95318602 15.40840566 15.64705269]';
    j1 = [0:5:35]';
    cfun_H1 = fit(j1, raw_H1, fit_func);

   BM1 = @(j) 0.0001*(pi*(cfun_H1.c/(1+cfun_H1.a*exp(-cfun_H1.b*j)))
   *(cfun_D1.c/(1+cfun_D1.a*exp(-cfun_D1.b*j)))^2*rho1)/(4*mu1);

   N1==symsum((cfun_D1.c/(1+cfun_D1.a*exp(-cfun_D1.b*j)))*N1*m1
   *(1-m1)^(j-1),j,1,T1-1)+(cfun_D1.c/(1+cfun_D1.a*exp(-cfun_D1.b*T1)))*N1*(1-m1)^(T1-1);
   m1 = vpasolve(eqn1,m1,0.00558);
   m1 = m1(imag(m1)==0 & real(m1)<1);

    n1 = zeros(200,200);
    CT1 = zeros(1,200);
    for j = 1:T1
        if j == T1
            n1(j,0+1) = N1*(1-m1)^(j-1);
        else
            n1(j,0+1) = N1*m1*(1-m1)^(j-1);
        end
    end
    for t = 1:100
        for j = 1:T1+t
            if j == 1
                n1(j,t+1) = N1*m1+sum(n1(P1-1:T1+t,t))*(1-m1)*p1;
            elseif j >= P1
                n1(j,t+1) = n1(j-1,t)*(1-m1)*(1-p1);
            else
                n1(j,t+1) = n1(j-1,t)*(1-m1);
            end
        end
        for j = 1:T1+t
            CT1(1,t) = CT1(1,t) + Kc*BM1(j)*n1(j,t+1);
        end
    end
    ... ...
    CT = CT1 + CT3 + CT3;
    
    CP1 = zeros(1,200);
    H1 = zeros(1,200);
    for t = 1:100
        for j = P1:t+T1
            H1(1,t+L1) = H1(1,t+L1)+n1(j-1,t)*(1-m1)*p1*BM1(j);
        end
        CP1(1,t) = Kc*(sum(alpha1*H1(1,t+L1-L1:t+L1))
        +sum(beta1*H1(1,t+L1-L2:t+L1))+sum(gamma1*H1(1,t+L1-L3:t+L1)));
        for j = 1:t+L1-L1
            CP1(1,t) = CP1(1,t) + Kc*alpha1*q*H1(1,j)*(exp(-r*(t-j-L1)));
        end
        for j = 1:t+L1-L2
            CP1(1,t) = CP1(1,t) + Kc*beta1*q*H1(1,j)
            *(exp(-r*(t-j-L2)));
        end
    end
    ... ...
    CP = CP1 + CP2 + CP3;
    
    C = CT + CP;
end


\end{lstlisting}


\subsection{B: Data}

\begin{table}[ht]
\centering
\caption{DBH and height variations of different tree species}
\begin{tabular}{lllllll} 
\hline
\multirow{2}{*}{Age [yr]} & \multicolumn{3}{c}{Diameter at breast-height [cm]} & \multicolumn{3}{c}{Height [m]}      \\
                          & Choerospondias & Taiwania & Toona                  & Choerospondias & Taiwania & Toona   \\ 
\hline
0                         & 0              & 0        & 0                      & 0              & 0        & 0       \\
5                         & 4.650          & 1.570    & 2.488                  & 6.216          & 2.507    & 4.798   \\
10                        & 9.913          & 9.211    & 5.155                  & 12.839         & 8.047    & 7.281   \\
15                        & 13.172         & 14.326   & 8.356                  & 14.101         & 13.630   & 10.664  \\
20                        & 14.926         & 15.505   & 12.194                 & 14.791         & 15.017   & 15.374  \\
25                        & 15.893         & 18.244   & 14.747                 & 14.953         & 17.427   & 19.125  \\
30                        & 16.699         & 20.540   & 15.900                 & 15.408         & 18.453   & 21.863  \\
35                        & 17.081         & 22.082   & 16.890                 & 15.647         & 20.347   & 23.282  \\
40                        & -              & 23.317   & 17.453                 & -              & 21.423   & 23.707  \\
\hline
\end{tabular}
\end{table}

\begin{table}[ht]
\centering
\begin{tabular}{llll} 
\hline
\multirow{2}{*}{Component} & \multicolumn{3}{l}{Biomass proportion [\%]}  \\
                           & Choerospondias & Taiwania & Toona            \\ 
\hline
Branches                   & 17.64          & 16.32    & 18.01            \\
Leaves                     & 7.66           & 6.49     & 7.51             \\
Root system                & 22.11          & 22.54    & 23.62            \\
Trunk with bark            & 52.59          & 54.65    & 50.86            \\
\hline
\end{tabular}
\end{table}

\begin{table}[ht]
\centering
\caption{Statistics of tree species at site}
\begin{tabular}{lllll}
\hline
               & \makecell{Number\\ density [hm2]} & \makecell{Average\\ BDH [cm]} & \makecell{Average\\ height [m]} & \makecell{Trunk\\ density [g/cm3]}  \\
\hline
Choerospondias & 1260                   & 16.86            & 15.65              & 0.596                    \\
Taiwania       & 1140                   & 23.01            & 21.27              & 0.358                    \\
Toona          & 870                    & 17.21            & 23.57              & 0.475                    \\
\hline
\end{tabular}
\end{table}